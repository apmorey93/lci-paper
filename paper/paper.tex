\documentclass[12pt]{article}

% ========= Packages =========
\usepackage{geometry}
\usepackage{setspace}
\usepackage{amsmath,amssymb,amsthm}
\usepackage{booktabs}
\usepackage{siunitx}
\usepackage{graphicx}
\usepackage{threeparttable}
\usepackage[round]{natbib}
\usepackage{hyperref}
\\usepackage{xcolor}
\\usepackage{textcomp}
\usepackage{tikz}
\usetikzlibrary{patterns,arrows.meta,positioning}
\usepackage[font=small,labelfont=bf,labelsep=period,skip=6pt]{caption}

\hypersetup{colorlinks=true,linkcolor=blue,citecolor=blue,urlcolor=blue}

% ========= Journal-like formatting =========
\geometry{margin=1in}
\doublespacing
\setlength{\parindent}{0.5in}
\pagestyle{plain}

% ---------- Page & float packing ----------
\raggedbottom
\clubpenalty=10000
\widowpenalty=10000
\displaywidowpenalty=10000
\renewcommand{\textfraction}{0.10}
\renewcommand{\topfraction}{0.90}
\renewcommand{\bottomfraction}{0.80}
\renewcommand{\floatpagefraction}{0.80}
\setcounter{topnumber}{3}
\setcounter{bottomnumber}{2}
\setcounter{totalnumber}{4}
\setlength{\textfloatsep}{10pt plus 2pt minus 2pt}
\setlength{\floatsep}{8pt plus 2pt minus 2pt}
\setlength{\intextsep}{8pt plus 2pt minus 2pt}
\setlength{\abovecaptionskip}{4pt}
\setlength{\belowcaptionskip}{0pt}

% ========= Math & units =========
\numberwithin{equation}{section}
\sisetup{
  detect-mode = true,
  table-number-alignment = center,
  retain-unity-mantissa = false,
  input-exponent-markers = e,
  group-separator = {,}
}
\DeclareSIUnit{\milliSecond}{ms}
\DeclareSIUnit{\dollar}{\$}
\DeclareSIUnit{\billion}{B}
\DeclareSIUnit{\percent}{\%}

% ========= Handy macros =========
\newcommand{\QOU}{\mathrm{QOU}}
\newcommand{\LCI}{\mathrm{LCI}}
\newcommand{\IPD}{\mathrm{IPD}}
\newcommand{\E}{\mathbb{E}}
\newcommand{\Prb}{\mathbb{P}}
\newcommand{\elloc}{\mathrm{loc}}

\begin{document}

% ================= Title page =================
\begin{center}
{\LARGE \textbf{Pricing Usable Intelligence: A Measurement Framework (LCI/IPD) with QoS Chance Constraints}}

\vspace{0.6cm}
Aditya Morey
\\[2pt]
\small{Your Affiliation Here}
\\[4pt]
\small{ORCID: \href{https://orcid.org/TODO}{TODO}}
\\[4pt]
\small{Corresponding author: \href{mailto:aditya.morey@institution.edu}{aditya.morey@institution.edu}; Tel: TODO}
\\[6pt]
\end{center}

\noindent\textbf{Abstract}

\noindent We define the \emph{Locational Cost of Intelligence} (LCI): the minimum cost to deliver one unit of task-equivalent AI output at specified accuracy, latency, reliability, and safety (QoS). LCI is a dual cost function under QoS \emph{chance constraints} that prices usable intelligencerather than tokensconditional on location-specific factor prices. We justify primitives with an engineering-aware \emph{serving model class} (GI/G/$k$ with processor sharing and batching), ground accuracy in neural scaling evidence, and recast theorems as \emph{properties} of the measurement object. A public-data calibration demonstrates quantitative implications, including a RAG--vs--scale efficiency region and the two-margin effect of transmission upgrades. We propose a chain Fisher \emph{Intelligence Price Deflator} (IPD) and provide a stylized location-choice mechanism linking transmission to both energy cost and latency.

\vspace{0.2cm}
\noindent\textbf{JEL Codes:} D24, L11, O33

\vspace{0.2cm}
\noindent\textbf{Keywords:} AI economics; hedonic price index; queueing; service level objectives; infrastructure policy

\newpage

% ================= 1. Introduction =================
\section{Introduction and Motivation}
The economics of AI hinge on \emph{usable} output under performance guarantees, not on raw tokens or list prices. We formalize the \emph{Locational Cost of Intelligence} (LCI) as the dual to a chance-constrained production problem: the cost-minimizing bundle of compute, power, labor, and networking that delivers one \emph{task-equivalent} unit at specified QoS with violation probability at most $\varepsilon$. Unlike \$/token metrics, LCI is (i) task-family specific, (ii) QoS-constrained, and (iii) location-aware.

\paragraph{Positioning.} We connect hedonic pricing of IT capital \citep{Triplett1989,Pakes2003,Byrne2017} with SLO-driven cloud operations and queueing, translating performance targets into a location-adjusted price of usable intelligence.

\paragraph{Notation.} Inputs $x=(H,P,W,N,O,\dots)$ denote hardware capacity, power, labor/wages, network, and orchestration/ops. Prices $p=(\kappa_K,c_E,w,c_N,\dots)$ are location specific. Load/utilization $u\in[0,u_{\max})$. Model scale $m$ (effective compute), retrieval depth $R$, and tooling $Z$ shape accuracy. QoS thresholds $(\bar a,\bar \ell,\bar q,\bar s)$ target accuracy $a$, latency $\ell$ (p95), reliability $q$, and safety $s$.

% ================= 2. Framework =================
\section{The LCI Framework}

\subsection{Quality-Adjusted Output (QOU)}
\begin{equation}
\QOU = T \cdot \phi(a,\ell,q,s),\qquad 
\phi(a,\ell,q,s)=\alpha(a)\,\lambda(\ell)\,\rho(q)\,\sigma(s),
\end{equation}
\begin{align}
\alpha(a)&=a^{\eta_a}, &
\rho(q)&=q^{\eta_q}, &
\sigma(s)&=s^{\eta_s},\\
\lambda(\ell)&=
\left(\frac{\bar\ell}{\max(\ell,\bar\ell)}\right)^{\eta_\ell}.
\end{align}
No penalty at or below the latency target $\bar\ell$; Appendix~B provides a smooth surrogate.

\subsection{Serving Model Class and Latency}
\begin{equation}
T = A_0\,H^{\beta_H} P^{\beta_P} W^{\beta_W} N^{\beta_N} O^{\beta_O}\cdot g(u),\qquad
g(u)=\left(1-\frac{u}{u_{\max}}\right)^{\gamma}.
\end{equation}
We use GI/G/$k$ with processor sharing (PS) and optional batching $(B,\tau_b)$. Let $W_\alpha$ be the $\alpha$-quantile of response time. We map $u\mapsto \ell=W_{0.95}$ and use heavy-traffic theory to show tail convexity near $u_{\max}$.

\paragraph{Accuracy from scaling.}
\begin{equation}
a = 1 - \exp\!\Big[-\big(\theta_m m^{\zeta_m} + \theta_R R^{\zeta_R} + \theta_{mR} m^{\zeta_m}R^{\zeta_R} + \theta_Z Z\big)\Big].
\end{equation}

\subsection{LCI with Chance Constraints}
\begin{equation}\label{eq:program}
\begin{aligned}
\min_{x,u,m,R,Z,B,\tau_b} \quad & C(x;p,\elloc)\\
\text{s.t.}\quad & \E[\QOU(x,u,m,R,Z)] \ge Q,\\
& \Prb\!\big(a\ge \bar a,\, \ell\le \bar \ell,\, q\ge \bar q,\, s\ge \bar s\big)\ \ge 1-\varepsilon.
\end{aligned}
\end{equation}
Define $\LCI = c_{\mathcal T}(Q;p,\elloc,\varepsilon)/Q$; by Shephard's lemma, $\partial \LCI/\partial p_j=x_j^*/Q$.

% ================= 3. Properties =================
\section{Properties}
\textbf{A. RAGScale Efficiency.}\;
\(
\frac{\theta_R \zeta_R R^{\zeta_R}}{MC_R/R} >
\frac{\theta_m \zeta_m m^{\zeta_m}}{MC_m/m}
\Rightarrow \text{RAG dominates.}
\)

\textbf{B. Two-Margin Transmission.}\; Transmission lowers $c_E$ and enables relocation that reduces network latency $\ell$.

\textbf{C. Convexity near saturation.}\; With batching $(B,\tau_b)$ and finite moments, $W_{0.95}(u)$ is convex near $u_{\max}$, so $\LCI(u)$ is convex.

% ================= 4. Calibration & Empirics =================
\section{Public-Data Calibration and Empirical Primitives}
\begin{table}[t]
\centering
\begin{threeparttable}
\caption{Empirical primitives (latest available at collection time)}
\label{tab:emp-primitives}
\begin{tabular}{l l l l}
\toprule
Category & Region / Item & Value & Source \\
\midrule
Industrial electricity & Virginia (US) & 9.49~\si{\cent\per\kilo\watt\hour} (Jul~2025) & EIA Table~5.6.A \\
Industrial electricity & Texas (US)    & 6.60~\si{\cent\per\kilo\watt\hour} (Jul~2025) & EIA Table~5.6.A \\
GPU (H100) price       & p5.4xlarge    & \$3.933 / accelerator-hour      & AWS Capacity Blocks \\
GPU (8$\times$H100)    & p5.48xlarge   & \$31.464 / instance-hour        & AWS Capacity Blocks \\
Retrieval (reads)      & Pinecone      & \$16 / 1M read units            & Pinecone Pricing \\
Retrieval (storage)    & Weaviate      & \$0.095 / 1M dims$\cdot$month   & Weaviate Pricing \\
Inter-region latency   & us-east-1$\to$eu-west-1 & p50 $\sim$70--90~\si{\milliSecond} & CloudPing \\
\bottomrule
\end{tabular}
\end{threeparttable}
\end{table}

% ================= 5. IPD =================
\section{Intelligence Price Deflator (IPD)}
\begin{equation}
\IPD_t \;=\; \prod_{\tau=1}^{t} \sqrt{
\sum_k s_{k,\tau-1}\,\frac{\LCI_{k,\tau}}{\LCI_{k,\tau-1}}
\cdot
\sum_k s_{k,\tau}\,\frac{\LCI_{k,\tau}}{\LCI_{k,\tau-1}}
}.
\end{equation}

% ================= 6. Limitations & Conclusion =================
\section{Limitations and Conclusion}
QoS fixed per period; multi-tenancy abstracted; task-family conditioning; functional-form risk handled via sensitivity. LCI prices \emph{usable} intelligence and supports firm strategy and policy evaluation.

% ====== Minimal References (to keep compile green) ======
\begin{thebibliography}{}
\bibitem[Byrne and Syverson(2017)]{Byrne2017} Byrne, D. M., and C. Syverson (2017), \emph{AER} 107(5): 168--172.
\bibitem[Pakes(2003)]{Pakes2003} Pakes, A. (2003), \emph{AER} 93(5): 1578--1596.
\bibitem[Triplett(1989)]{Triplett1989} Triplett, J. E. (1989), \emph{Brookings Papers on Economic Activity, Microeconomics}: 373--438.
\end{thebibliography}

\end{document}

